\documentclass{article}
\usepackage[spanish]{babel}
\usepackage{amsmath, amssymb}
\usepackage{lipsum}
\usepackage{graphicx}
\usepackage{xcolor}

% Agreguemos el paquete array
\usepackage{array}

\title{CDE \LaTeX}
\author{Clase 5}
\date{27 de marzo del 2023}

\begin{document}

\maketitle

\begin{abstract}
	\lipsum[2]
\end{abstract}

\section{El entorno tabular}
\noindent Sintaxis

\begin{verbatim}
\begin{tabular}[posicion]{Definicion}
a_{11} & a_{12} & a_{13} \\
a_{21} & a_{22} & a_{23}
\end{tabular}
\end{verbatim}

Profundicemos en el argumento \verb*|Definicion|:
Definiremos el numero de columnas y la justificación de cada una de esas columnas.
\begin{itemize}
\item Usar las letras \verb*|l| (left), \verb*|c| (center) y \verb*|r| (right) para justificar el texto dentro de cada columna
\begin{verbatim}
	\begin{tabular}[]{lrrc}
		a_{11} & a_{12} & a_{13} & a_{14}\\
		a_{21} & a_{22} & a_{23} & a_{24}
	\end{tabular}
\end{verbatim}

Veamos un ejemplo :

\begin{tabular}[]{lrrc}
	$a_{11}$ & $a_{12}$ & $a_{13}$ & $a_{14}$\\
	$a_{21}$ & $a_{22}$ & $a_{23}$ & $a_{24}$
\end{tabular}

Veamos otro ejemplo :

\begin{tabular}[]{lrrc}
	$a_{11}$ & $a_{12}$ & $a_{13}$ & $a_{14}$\\
	$a_{21}$ & $a_{22}$ & $a_{23}$ & \lipsum[2]
\end{tabular}


\item Definamos un párrafo como elemento de un \verb*|tabular|

Veamos un ejemplo : 

\begin{tabular}[]{lrrp{9cm}}
	$a_{11}$ & $a_{12}$ & $a_{13}$ & $a_{14}$\\
	$a_{21}$ & $a_{22}$ & $a_{23}$ & \lipsum[2]
\end{tabular}
\end{itemize}

\section{Agreguemos lineas horizontales y verticales a tabular}

\noindent Sintaxis

\begin{verbatim}
	\begin{tabular}[posicion]{Definicion}
		a_{11} & a_{12} & a_{13} \\
		a_{21} & a_{22} & a_{23}
	\end{tabular}
\end{verbatim}

Notar el uso de \verb*|hline| y la barra vertical

\begin{verbatim}
	\begin{tabular}[]{|c|c|c|}
		\hline
		a_{11} & a_{12} & a_{13} \\
		\hline \\
		\hline
		a_{21} & a_{22} & a_{23}
		\hline
	\end{tabular}
\end{verbatim}

En ejecución :
\begin{center}
\begin{tabular}[]{|c|c|c|}
	\hline
	$a_{11}$ & $a_{12}$ & $a_{13}$ \\
	\hline 
	$a_{21}$ & $a_{22}$ & $a_{23}$ \\
	\hline
	$a_{21}$ & $a_{22}$ & $a_{23}$ \\
	\hline
	$a_{21}$ & $a_{22}$ & $a_{23}$ \\
	\hline
	$a_{21}$ & $a_{22}$ & $a_{23}$ \\
	\hline
\end{tabular}
\end{center}

\subsection{cline}
El comando \verb*|cline{i-j}| trazara la linea horizontal entre las columnas de la \verb*|i| a la \verb*|j|.
\begin{center}
	\begin{tabular}{cccr}
		\hline
		Producto & Precio Unitario & Cantidad & Subtotal \\
		\hline
		Monitores & 1250 & 2 & 2500\\
				  & 2780 & 2 & 5560\\
				  & 16800 & 4 & 67200\\
		\hline
		GPU		  & 2500 & 2 & 5000\\
				  & 3700 & 3 & 11100\\
				  \cline{4-4}
				  &      &   & 78300
	\end{tabular}
\end{center}

\subsection{Encolumnar el punto decimal}
Consideremos que tenemos 4 números punto flotante, donde cada uno de estos números posee una distinta cantidad de cifras en la mantisa : 3.14159, 2.71, 123.4567, 0. 

\begin{center}
	\begin{tabular}{cl}
		\hline
		Var1 & 3.14159\\
		Var2 & 2.71\\
		Var3 & 123.4567\\
		Var4 & 0\\
		\hline
	\end{tabular}
\end{center}

Consideremos arreglar/encolumnar la ubicación del punto decimal . 

\begin{center}
	\begin{tabular}{cr@{.}l}
		Var & Numeritos & \\
		\hline
		Var1 & 3&14159\\
		Var2 & 2&71\\
		Var3 & 123&4567\\
		Var4 & 0&\\
		\hline
	\end{tabular}
\end{center}

\section{Construcción del indice de figuras}
\lipsum[2-5]

\begin{figure}[h]
	\centering
	\includegraphics[width=0.6\textwidth]{Figuras/deepma.jpg}
	\caption{Función de Costo}
	\label{fig: Costo}
\end{figure}
\lipsum[1-9]

\begin{figure}[h]
	\centering
	\includegraphics[width=0.6\textwidth]{Figuras/gausskl.jpg}
	\caption{El Príncipe : Gauss}
	\label{fig: gauss}
\end{figure}
\lipsum[3]

\begin{figure}[h]
	\centering
	\includegraphics[width=0.6\textwidth]{Figuras/descarga.jpg}
	\caption{Julio Ramon Ribeyro}
	\label{fig: Maestro}
\end{figure}

\lipsum[3-7]
\begin{figure}[h]
	\centering
	\includegraphics[width=0.6\textwidth]{Figuras/Data-science-without-mathematics.jpg}
	\caption{Red Neuronal}
	\label{fig: perceptron}
\end{figure}

\section{Construcción del indice de tablas}
\lipsum[15]
\begin{table}[h]
	\centering
	\begin{tabular}{cccr}
		\hline
		Producto & Precio Unitario & Cantidad & Subtotal \\
		\hline
		Monitores & 1250 & 2 & 2500\\
		& 2780 & 2 & 5560\\
		& 16800 & 4 & 67200\\
		\hline
		GPU		  & 2500 & 2 & 5000\\
		& 3700 & 3 & 11100\\
		\cline{4-4}
		&      &   & 78300
	\end{tabular}
\caption{Costo de monitor y Co-Procesador Matematico}
\label{table:1}
\end{table}

\lipsum[5-9]

\begin{table}[h]
	\centering
	\begin{tabular}{cccr}
		\hline
		Producto & Precio Unitario & Cantidad & Subtotal \\
		\hline
		Monitores & 1250 & 2 & 2500\\
		& 2780 & 2 & 5560\\
		& 16800 & 4 & 67200\\
		\hline
		GPU		  & 2500 & 2 & 5000\\
		& 3700 & 3 & 11100\\
		\cline{4-4}
		&      &   & 78300
	\end{tabular}
	\caption{Costo de monitor y Co-Procesador Matematico - Version 2}
	\label{table:2}
\end{table}

\lipsum[11-14]

\begin{table}[h]
	\centering
	\begin{tabular}{cccr}
		\hline
		Producto & Precio Unitario & Cantidad & Subtotal \\
		\hline
		Monitores & 1250 & 2 & 2500\\
		& 2780 & 2 & 5560\\
		& 16800 & 4 & 67200\\
		\hline
		GPU		  & 2500 & 2 & 5000\\
		& 3700 & 3 & 11100\\
		\cline{4-4}
		&      &   & 78300
	\end{tabular} 
	\caption{Costo de monitor y Co-Procesador Matematico - Tablita 3}
	\label{table:3}
\end{table}

FIN.

\newpage
\listoftables
\listoffigures
\tableofcontents

\end{document}