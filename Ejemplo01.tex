\documentclass{article}
\usepackage[spanish]{babel}
\usepackage{amsmath, amssymb}
\usepackage{lipsum}
\usepackage{graphicx}
\usepackage{xcolor}

% Agreguemos el paquete array
\usepackage{array}

\title{CDE \LaTeX}
\author{Clase 5}
\date{27 de marzo del 2023}

\begin{document}

\maketitle

\begin{abstract}
	\lipsum[2]
\end{abstract}

\section{El entorno tabular}
\noindent Sintaxis

\begin{verbatim}
\begin{tabular}[posicion]{Definicion}
a_{11} & a_{12} & a_{13} \\
a_{21} & a_{22} & a_{23}
\end{tabular}
\end{verbatim}

Profundicemos en el argumento \verb*|Definicion|:
Definiremos el numero de columnas y la justificación de cada una de esas columnas.
\begin{itemize}
\item Usar las letras \verb*|l| (left), \verb*|c| (center) y \verb*|r| (right) para justificar el texto dentro de cada columna
\begin{verbatim}
	\begin{tabular}[]{lrrc}
		a_{11} & a_{12} & a_{13} & a_{14}\\
		a_{21} & a_{22} & a_{23} & a_{24}
	\end{tabular}
\end{verbatim}

Veamos un ejemplo :

\begin{tabular}[]{lrrc}
	$a_{11}$ & $a_{12}$ & $a_{13}$ & $a_{14}$\\
	$a_{21}$ & $a_{22}$ & $a_{23}$ & $a_{24}$
\end{tabular}

Veamos otro ejemplo :

\begin{tabular}[]{lrrc}
	$a_{11}$ & $a_{12}$ & $a_{13}$ & $a_{14}$\\
	$a_{21}$ & $a_{22}$ & $a_{23}$ & \lipsum[2]
\end{tabular}


\item Definamos un párrafo como elemento de un \verb*|tabular|

Veamos un ejemplo : 

\begin{tabular}[]{lrrp{9cm}}
	$a_{11}$ & $a_{12}$ & $a_{13}$ & $a_{14}$\\
	$a_{21}$ & $a_{22}$ & $a_{23}$ & \lipsum[2]
\end{tabular}




\end{itemize}




\end{document}